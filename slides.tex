\documentclass{beamer}
\usepackage{fontspec}
\usepackage{minted}
\usepackage{tango}

\mode<presentation>{\usetheme{Copenhagen}}
\title{Using static typing features for fun and profit}
\author{Marek~Kubica}
\date{6.~August~2013}
\institute{Lambda Munich}

\setsansfont{Yanone Kaffeesatz Regular}
\setmainfont{EB Garamond}
\setmonofont[Scale=0.8]{Droid Sans Mono Dotted}

\usemintedstyle{tango}
\usecolortheme{rwo}
\setbeamerfont{title}{family=\rmfamily}
%\setbeamertemplate{blocks}[rounded=false]
%\useinnertheme{default}
\setbeamertemplate{title page}[default][colsep=-0bp,rounded=false,shadow=\beamer@themerounded@shadow]

%% get rid of navigation symbols
%\setbeamertemplate{navigation symbols}{}
\beamertemplatenavigationsymbolsempty

\begin{document}

\frame{\titlepage}

\begin{frame}{Some words upfront}
  \begin{alertblock}{Disclaimer}
    I am not a professional OCaml programmer, type theoretician etc.
  \end{alertblock}
  \pause
  \begin{alertblock}{Static typing?}
    I am neither a static typing weenie, I like dynamically typed languages
    too, so put your flamewar away (for now).
  \end{alertblock}
\end{frame}

\begin{frame}[fragile]{Static typing the C way}
  Let's check how C handles look like. How about checking libarchive.
  \begin{minted}{c}
    __LA_DECL struct archive* archive_read_new(void);
    __LA_DECL struct archive* archive_write_new(void);
  \end{minted}
  \begin{itemize}
    \item Opaque pointer to some struct
    \item Write handles and read handles have the same type
  \end{itemize}
\end{frame}

\begin{frame}
  So, what can we do with these handles?
  \pause
  \begin{itemize}
    \item Create them
    \item Open them
    \item Read from them
    \item Write from them
    \item Close them
  \end{itemize}
  \pause
  Cool.
  \pause
  \alert{But what if we screw up?}
\end{frame}

\end{document}
